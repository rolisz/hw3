\documentclass[a4paper,12pt]{article}
\usepackage{xltxtra}
\usepackage{fancyhdr}
\usepackage[top=1in, bottom=1.5in, left=1cm, right=1cm]{geometry}
\usepackage{setspace}
\onehalfspacing
% Chestiile pentru mate
\usepackage{amsmath}
\usepackage{amsfonts}
\usepackage{bbm}
\author{Roland Szabo, gr. 235}
\usepackage{fancyhdr}
\pagestyle{fancy}
\makeatletter
\makeatother
\lhead{Roland Szabo, gr. 235}
\rhead{Assignment 3, 20.11.13}
\begin{document}

Test the primality of 2887 and 481 using the Miller-Rabin test. Check with 3 bases if necessary. 

For $ n = 2887 $.

We write $ n - 1 = 2^s t $:

$ 2886 = 2^1 * 1443 \Rightarrow s = 1, t = 1443 $

We choose $ b = 13 $.

$ r = b^t \mod n = 13^1443 = 13^{2^0 + 2^1 + 2^5 + 2^7 + 2^8 + 2^10} = 13 \cdot 13^{2^1 + 2^5 + 2^7 + 2^8 + 2^10} = 13 \cdot 13^2 \cdot 13^{2^5 + 2^7 + 2^8 + 2^10} = 2197 \cdot 13^32 \cdot 13^{2^7 + 2^8 + 2^10} = 1422 \cdot 1422^4 \cdot 1422^8 \cdot 13^1023 = 1 $

Because $ r = 1 $, 2887 has passed the first iteration of the Miller-Rabin test

We choose $ b = 7 $

$ r = b^t \mod n = 7^1443 = 7^{2^0 + 2^1 + 2^5 + 2^7 + 2^8 + 2^10} = 7 \cdot 7^{2^1 + 2^5 + 2^7 + 2^8 + 2^10} = 7 \cdot 7^2 \cdot 7^{2^5 + 2^7 + 2^8 + 2^10} = 343 \cdot 13^32 \cdot 7^{2^7 + 2^8 + 2^10} = ... = 1 \mod 2887$

Because $ r = 1 $, 2887 has passed the second iteration of the Miller-Rabin test

We choose $ b = 69 $

$ r = b^t \mod n = 6^1443 \mod 2887 = 1 \mod 2887$

Because $ r = 1 $, 2887 has passed the third iteration of the Miller-Rabin test, and we can say that it is prime with probability $ 1 - 1/4^3 = 98.43 \%$.

For $ n =  481 $.

$ 480 = 2^5 \cdot 15 \Rightarrow s = 5, t = 15 $. 

We choose $ b = 432 $.

$ r = b^t \mod n = 432^15 \mod 481 = 432^{2^0 + 2^1 + 2^2 + 2^3} \mod 481 = 27 \mod 481 $. 

Because $ 27 \neq 1 \text{ and } 27 \neq 480 $ we repeatedly square r, $ s - 1 $ times, and we get: $ 248, 417, 248, 417 $. Because the last one is not $ n-1 = 480 $, we can say that 481 is composite. 

\end{document}