\documentclass[a4paper,12pt]{article}
\usepackage{xltxtra}
\usepackage{fancyhdr}
\usepackage[top=1in, bottom=1.5in, left=1cm, right=1cm]{geometry}
\usepackage{setspace}
\onehalfspacing
% Chestiile pentru mate
\usepackage{amsmath}
\usepackage{amsfonts}
\usepackage{bbm}
\author{Roland Szabo, gr. 235}
\usepackage{fancyhdr}
\pagestyle{fancy}
\makeatletter
\makeatother
\lhead{Roland Szabo, gr. 235}
\rhead{Assignment 1, 23.10.13}
\begin{document}


We use an alphabet composed of a blank space and the 26 letters of the English alphabet.


$ n = 27, m = 2 $


We will use as key the assignment number and current date: $1, 23, 10, 13$


$$ K = \begin{pmatrix}
1 & 23 \\
10 & 13
\end{pmatrix}$$


The plaintext is SZABO, which is numerically: $19, 26, 1, 2, 15 $. Because it has
an length that is not a multiple of $ m $, we pad it with a space character.


$ P = (19, 26), (1, 2), (15, 0) $


To encrypt we multiply modulo 27 $ K $ with each group of two letters from the plaintext:


$$ \left[\begin{pmatrix}
19 & 26
\end{pmatrix}
\begin{pmatrix}
1 & 23 \\
10 & 13
\end{pmatrix}
=
\begin{pmatrix}
9 \\
19
\end{pmatrix},
\begin{pmatrix}
1 & 2
\end{pmatrix}
\begin{pmatrix}
1 & 23 \\
10 & 13
\end{pmatrix}
=
\begin{pmatrix}
21 \\
22
\end{pmatrix},
\begin{pmatrix}
15 & 0
\end{pmatrix}
\begin{pmatrix}
1 & 23 \\
10 & 13
\end{pmatrix}
=
\begin{pmatrix}
15 \\
21
\end{pmatrix}\right]$$


Converting $9, 19, 21, 22, 15, 21 $ to our alphabet, we get that the ciphertext is ISUVOU.


For decryption we need the decryption key, which is the inverse of our encryption key. Using Cramer's rule:

$$ K^{-1} = (1 \cdot 13 - 23 \cdot 10)^{-1}\begin{pmatrix}
13&-23\\
-10 & 1
\end{pmatrix} = 26 \begin{pmatrix}
13&-23\\
-10 & 1
\end{pmatrix} = \begin{pmatrix}
14 & 23\\
10 & 26
\end{pmatrix} $$

To decrypt the cyphertext, we multiply the decryption key with each group of two letters from the cypertext:

$$ \left[\begin{pmatrix}
9 & 19
\end{pmatrix}
\begin{pmatrix}
14 & 23\\
10 & 26
\end{pmatrix}
=
\begin{pmatrix}
19 \\
26
\end{pmatrix},
\begin{pmatrix}
21 & 22
\end{pmatrix}
\begin{pmatrix}
14 & 23\\
10 & 26
\end{pmatrix}
=
\begin{pmatrix}
1 \\
2
\end{pmatrix},
\begin{pmatrix}
15 & 21
\end{pmatrix}
\begin{pmatrix}
14 & 23\\
10 & 26
\end{pmatrix}
=
\begin{pmatrix}
15 \\
0
\end{pmatrix}\right]$$

The result is SZABO\_, which is what we encrypted in the first place, with the trailing space. 
\end{document}